\documentclass[10pt]{article}

% This is a helpful package that puts math inside length specifications
\usepackage{calc}


% Simpler bibsection for CV sections
% (thanks to natbib for inspiration)
\makeatletter
\newlength{\bibhang}
\setlength{\bibhang}{1em}
\newlength{\bibsep}
 {\@listi \global\bibsep\itemsep \global\advance\bibsep by\parsep}
\newenvironment{bibsection}%
        {\vspace{-\baselineskip}\begin{list}{}{%
       \setlength{\leftmargin}{\bibhang}%
       \setlength{\itemindent}{-\leftmargin}%
       \setlength{\itemsep}{\bibsep}%
       \setlength{\parsep}{\z@}%
        \setlength{\partopsep}{0pt}%
        \setlength{\topsep}{0pt}}}
        {\end{list}\vspace{-.6\baselineskip}}
\makeatother

% Layout: Puts the section titles on left side of page
\reversemarginpar

%
%         PAPER SIZE, PAGE NUMBER, AND DOCUMENT LAYOUT NOTES:
%
% The next \usepackage line changes the layout for CV style section
% headings as marginal notes. It also sets up the paper size as either
% letter or A4. By default, letter was used. If A4 paper is desired,
% comment out the letterpaper lines and uncomment the a4paper lines.
%
% As you can see, the margin widths and section title widths can be
% easily adjusted.
%
% ALSO: Notice that the includefoot option can be commented OUT in order
% to put the PAGE NUMBER *IN* the bottom margin. This will make the
% effective text area larger.
%
% IF YOU WISH TO REMOVE THE ``of LASTPAGE'' next to each page number,
% see the note about the +LP and -LP lines below. Comment out the +LP
% and uncomment the -LP.
%
% IF YOU WISH TO REMOVE PAGE NUMBERS, be sure that the includefoot line
% is uncommented and ALSO uncomment the \pagestyle{empty} a few lines
% below.
%

%% Use these lines for letter-sized paper
\usepackage[paper=letterpaper,
            %includefoot, % Uncomment to put page number above margin
            marginparwidth=1.10in,     % Length of section titles
            marginparsep=.1in,       % Space between titles and text
            margin=.5in,               % 1 inch margins
            includemp]{geometry}

%% Use these lines for A4-sized paper
%\usepackage[paper=a4paper,
%            %includefoot, % Uncomment to put page number above margin
%            marginparwidth=30.5mm,    % Length of section titles
%            marginparsep=1.5mm,       % Space between titles and text
%            margin=25mm,              % 25mm margins
%            includemp]{geometry}

%% More layout: Get rid of indenting throughout entire document
\setlength{\parindent}{0in}

%% This gives us fun enumeration environments. compactitem will be nice.
\usepackage{paralist}

%% Reference the last page in the page number
%
% NOTE: comment the +LP line and uncomment the -LP line to have page
%       numbers without the ``of ##'' last page reference)
%
% NOTE: uncomment the \pagestyle{empty} line to get rid of all page
%       numbers (make sure includefoot is commented out above)
%
\usepackage{fancyhdr,lastpage}
\pagestyle{fancy}
\pagestyle{empty}      % Uncomment this to get rid of page numbers
\fancyhf{}\renewcommand{\headrulewidth}{0pt}
\fancyfootoffset{\marginparsep+\marginparwidth}
\newlength{\footpageshift}
\setlength{\footpageshift}
          {0.5\textwidth+0.5\marginparsep+0.5\marginparwidth-2in}
\lfoot{\hspace{\footpageshift}%
       \parbox{4in}{\, \hfill %
                    \arabic{page} of \protect\pageref*{LastPage} % +LP
%                    \arabic{page}                               % -LP
                    \hfill \,}}

% Finally, give us PDF bookmarks
\usepackage{color,hyperref}
\definecolor{darkblue}{rgb}{0.0,0.0,0.3}
\hypersetup{colorlinks,breaklinks,
            linkcolor=darkblue,urlcolor=darkblue,
            anchorcolor=darkblue,citecolor=darkblue}

%%%%%%%%%%%%%%%%%%%%%%%% End Document Setup %%%%%%%%%%%%%%%%%%%%%%%%%%%%


%%%%%%%%%%%%%%%%%%%%%%%%%%% Helper Commands %%%%%%%%%%%%%%%%%%%%%%%%%%%%

% The title (name) with a horizontal rule under it
%
% Usage: \makeheading{name}
%
% Place at top of document. It should be the first thing.
\newcommand{\makeheading}[1]%
        {\hspace*{-\marginparsep minus \marginparwidth}%
         \begin{minipage}[t]{\textwidth+\marginparwidth+\marginparsep}%
                {\large \bfseries #1}\\[-0.15\baselineskip]%
                 \rule{\columnwidth}{0.75pt}%
         \end{minipage}}

% The section headings
%
% Usage: \section{section name}
%
% Follow this section IMMEDIATELY with the first line of the section
% text. Do not put whitespace in between. That is, do this:
%
%       \section{My Information}
%       Here is my information.
%
% and NOT this:
%
%       \section{My Information}
%
%       Here is my information.
%
% Otherwise the top of the section header will not line up with the top
% of the section. Of course, using a single comment character (%) on
% empty lines allows for the function of the first example with the
% readability of the second example.
\renewcommand{\section}[2]%
        {\pagebreak[2]\vspace{1.3\baselineskip}%
         \phantomsection\addcontentsline{toc}{section}{#1}%
         \hspace{0in}%
         \marginpar{
         \raggedright \scshape #1}#2}

% An itemize-style list with lots of space between items
\newenvironment{outerlist}[1][\enskip\textbullet]%
        {\begin{itemize}[#1]}{\end{itemize}%
         \vspace{-.6\baselineskip}}

% An environment IDENTICAL to outerlist that has better pre-list spacing
% when used as the first thing in a \section
\newenvironment{lonelist}[1][\enskip\textbullet]%
        {\vspace{-\baselineskip}\begin{list}{#1}{%
        \setlength{\partopsep}{0pt}%
        \setlength{\topsep}{0pt}}}
        {\end{list}\vspace{-.6\baselineskip}}

% An itemize-style list with little space between items
\newenvironment{innerlist}[1][\enskip\textbullet]%
        {\begin{compactitem}[#1]}{\end{compactitem}}

% An environment IDENTICAL to innerlist that has better pre-list spacing
% when used as the first thing in a \section
\newenvironment{loneinnerlist}[1][\enskip\textbullet]%
        {\vspace{-\baselineskip}\begin{compactitem}[#1]}
        {\end{compactitem}\vspace{-.6\baselineskip}}

% To add some paragraph space between lines.
% This also tells LaTeX to preferably break a page on one of these gaps
% if there is a needed pagebreak nearby.
\newcommand{\blankline}{\quad\pagebreak[2]}

% Uses hyperref to link DOI
\newcommand\doilink[1]{\href{http://dx.doi.org/#1}{#1}}
\newcommand\doi[1]{doi:\doilink{#1}}

% For \url{SOME_URL}, links SOME_URL to the url SOME_URL
\providecommand*\url[1]{\href{#1}{#1}}
% Same as above, but pretty-prints SOME_URL in teletype fixed-width font
\renewcommand*\url[1]{\href{#1}{\texttt{#1}}}

% For \email{ADDRESS}, links ADDRESS to the url mailto:ADDRESS
\providecommand*\email[1]{\href{mailto:#1}{#1}}
% Same as above, but pretty-prints ADDRESS in teletype fixed-width font
%\renewcommand*\email[1]{\href{mailto:#1}{\texttt{#1}}}

%%%%%%%%%%%%%%%%%%%%%%%% End Helper Commands %%%%%%%%%%%%%%%%%%%%%%%%%%%

%%%%%%%%%%%%%%%%%%%%%%%%% Begin CV Document %%%%%%%%%%%%%%%%%%%%%%%%%%%%

\begin{document}
\makeheading{Darren M. Jennings}

\section{Contact Information}
%
% NOTE: Mind where the & separators and \\ breaks are in the following
%       table.
%
% ALSO: \rcollength is the width of the right column of the table
%       (adjust it to your liking; default is 1.85in).
%
\newlength{\rcollength}\setlength{\rcollength}{2.5in}%
%
\begin{tabular}[t]{@{}p{\textwidth-\rcollength}p{\rcollength}}
2375 18th Ave &       \textit{Mobile:} +1-502-544-4320 \\
San Francisco, CA 94116 & \textit{E-mail:} \email{dmjenn02@gmail.com} \\
\end{tabular}


%
%
\section{Education}
\href{http://www.louisville.edu/}{\textbf{The University of Louisville}},
Louisville, KY USA
\begin{outerlist}
\item[]
        \href{http://www.physics.louisville.edu/}
             {\textit{B.S. Professional Physics}}
                                  \hfill \textbf{2011}             
        \begin{innerlist}
        \item Minor Field of Study: Mathematics
        \end{innerlist}
\end{outerlist}

\blankline

\href{http://www.louisville.edu/}{\textbf{The University of Louisville}},
Louisville, KY USA
\begin{outerlist}
\item[]
        \href{http://louisville.edu/speed/computer/}
             {\textit{M.S. Computer Engineering Computer Science (CECS)}}
                     \hfill \textbf{Fall 2012 to Fall 2014}
                             \begin{innerlist}
        \item Degree not completed
        \item Courses completed: CECS 130 \textit{Intro to Programming Languages}, CECS 302 \textit{Data Structures}, CECS 310 \textit{Discrete Structures}, CECS 590 {\textit{Mobile Programming for iOS Developers}}, CECS 420 {\textit{Design of Operating Systems}}
        \end{innerlist}
\end{outerlist}

%
%
%
\section{Work Experience}
%
%
%
\href{https://www.educenets.com}{\textbf{Educents}},
San Francisco, CA
\begin{outerlist}
\item[] \textit{Software Engineer}%
        \hfill \textbf{June 2016 to present}
\begin{innerlist}
\item Worked on Educents.com, a Magento e-commerce Marketplace, building customized systems to support Education through unique learning material.
\item Overhauled the Educents Seller dashboard and onboarding codebase migrating javascript from jQuery to Vue.js speeding up develoment time, improving code quality, and increasing conversion of Sellers through onboarding. Displayed robust charting giving actionable data to sellers ingested from HEAP via an ETL process.
\item Introduced Engineering team's first automated unit and acceptance tests which improved code integrity and allowed us to catch bugs faster.
\end{innerlist}
\end{outerlist}
\blankline

\href{http://www.generationtux.com}{\textbf{Generation Tux}} / \href{http://www.ztailors.com}{\textbf{zTailors}},
San Francisco, CA
\begin{outerlist}
\item[] \textit{Software Engineer}%
        \hfill \textbf{November 2015 to present}
\begin{innerlist}
\item Led a team of engineers building technical solutions for the first national on-demand tailoring platform building APIs and Dashboards in PHP Symfony 3. 
\item Primary developer on the \href{https://www.ztailors.com/garments}{zWidget}, an embeddable Javascript widget (React JS) for B2B partners to bring tailor booking to any site. Successfully replaced legacy php Symfony forms on \href{http://www.ztailors.com/garments}{zTailors.com} in favor of a lighter weight javascript single page application (SPA).
\item Developed report generation through scheduled tasks executed on Spotify's python Luigi module.
\item Worked to maintain code quality through unit/acceptance testing, code reviews, and continuous integration.
\end{innerlist}
\end{outerlist}
\blankline

\href{https://www.genscape.com}{\textbf{Genscape Inc.}},
Louisville, KY USA
\begin{outerlist}
\item[] \textit{Software Developer}%
        \hfill \textbf{May 2014 to November 2015}
\begin{innerlist}
\item Worked independently on Genscape's oil pipeline flow modeling calibration process. Designed and developed ``Excalibrator'', a responsive SPA which instruments complex data generation and analysis workflows cutting process times in half. Using AngularJS 1.4x on the front-end and PostgreSQL/SQL Server/Oracle on the data tier along with a suite of RESTful Web API services on the backend.
\item Primary developer alongside an offshore development team creating Genscape's internal tools platform which served as an ecosystem to facilitate faster app development/integration for Genscape devs through consolidated authentication and application management.
\item Developed extensively on RESTful Web API services and a mixture of Java and .NET (ASP.NET MVC) webapps for both customer facing and internal toolsets.
\item Worked with and managed a myriad of legacy software using various technologies, mostly in Java, Groovy/Grails to help overhaul and transition to .NET.
\item Earned a "Spot Award" in January 2014 for taking the initiative to develop a solution for image analysis pain points that plagued multiple groups in the organization. Built tool specifications and developed alone. The application received wide-spread adoption and is integral to Genscape image analysis.
\end{innerlist}
\end{outerlist}
\blankline

\begin{outerlist}
\item[] \textit{Analyst / Software Developer}%
        \hfill \textbf{October 2011 to 2014}
\begin{innerlist}
\item Worked with a team of analysts to provide real-time data on the crude oil/NGL supply chain, specifically transportation (pipelines, rail, ship) data using Genscape's proprietary field monitors. Included analysis of electromagnetic, infra-red, and visual spectrum monitor data, field research of energy infrastructure, publication of intelligence reports/alerts, and software development for internal tools used by Genscape analysts.
\end{innerlist}
\end{outerlist}
\blankline

\href{http://www.roastwatchapp.com}{\textbf{Darren Jennings}},
Louisville, KY USA
\begin{outerlist}
\item[] \textit{Independent Web/iOS Software Developer}%
        \hfill \textbf{October 2013 to present}
\begin{innerlist}
\item Developed RoastWatch (\href{http://www.roastwatchapp.com/}{http://www.roastwatchapp.com/}), an iOS tool for roasting coffee. Worked alone on the project from concept to the App store. Released in the App Store April 2014. Developed in Obj-C.
\item Worked as an independent contractor developing apps using both Phonegap/Cordova and also the Ionic framework.
\end{innerlist}
\end{outerlist}
\blankline
% \clearpage

\href{http://www.louisville.edu}{\textbf{The University of Louisville}},
Louisville, KY USA
\begin{outerlist}
\item[] \textit{Undergraduate Researcher}%
        \hfill \textbf{October 2008 to 2011}
\begin{innerlist}
\item Employed in a High Energy Physics (HEP) analysis group researching baryon production in $e^-e^+$ collisions at the SLAC National AcceleratorLaboratory as part of the BaBar Collaboration. Worked with large datasets developing applications using C++ data mining and worked with ROOT, a C++ analysis framework.
\item Developed an independent analysis studying particle event topology as part of a larger exploration of QCD physics in inclusive baryon production.
\end{innerlist}
\end{outerlist}
%
%
%
\section{Technical Skills}
{\textbf{Languages Proficient}}: PHP (Symfony 3, Laraval, Magento), C{\#}, Java, Javascript.\\
{\textbf{Languages Experienced}}: Objective-C, Python, C$+$$+$, Groovy\\
{\textbf{Databases}}: PostgreSQL, MySQL, Microsoft SQL Server. SQLite, Oracle\\

%
%
%
\section{References}
\textbf{Ron Peled}
phone: ~+1-415-516-7777
\begin{innerlist}
    \item \emph{Ron was CTO at Educents and a mentor.}
\end{innerlist}
\textbf{Matt Howland}
phone: ~+1-415-420-2976
\begin{innerlist}
    \item \emph{I reported directly to Matt who was CTO and managed engineering at Generation Tux / zTailors}
\end{innerlist}
\textbf{Eddie Tinsley}
phone: ~+1-502-876-6169
\begin{innerlist}
    \item \emph{I worked closely with Eddie at Genscape transitioning legacy systems and driving new development for Genscape's Oil product.}
\end{innerlist}

%
\blankline

\end{document}

%%%%%%%%%%%%%%%%%%%%%%%%%% End CV Document %%%%%%%%%%%%%%%%%%%%%%%%%%%%%